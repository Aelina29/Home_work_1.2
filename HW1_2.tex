%формат
\documentclass[a4paper,25pt]{book}
%русс яз и кодировки
\usepackage[T2A]{fontenc}
\usepackage[utf8]{inputenc}
\usepackage[english,russian]{babel}
%поля
\usepackage[left=3cm,right=3cm,
    top=2cm,bottom=2cm]{geometry}
%матеша
\usepackage{amsmath,amsfonts,amssymb,amsthm,mathtools}
%заголовок
\title{Теория алгоритмов

Домашнее задание 1

Часть 1}
\author{Кондратьева Аэлина 2.9}
\date{\today}

%документ
\begin{document}
\maketitle
\newpage

\textbf{(И) Вычисление частного и остатка от деления числа, 
заданного в унарной системесчисления, на два (над алфавитом
$\Sigma = \{1,\#\}$). Результат должен записыватьсяв виде 
частное$\#$остаток.
Ноль должен соответствовать пустому слову.}\\\\
Задан алфавит $\Sigma = \{1,\#\}$\\\\
$ \left\{
\begin{aligned}
& \#11 \rightarrow 1\# \\
&  \epsilon \rightarrow \# \\
\end{aligned}
\right.$
\\\\
Примеры:\\
1 ) $1 \Rightarrow \#1$\\
2 ) $11 \Rightarrow \#11 \Rightarrow 1\# $\\
3 ) $111 \Rightarrow \#111 \Rightarrow 1\#1 $\\
4 ) $1111 \Rightarrow \#1111 \Rightarrow 1\#11 \Rightarrow 11\# $\\
5 ) $11111 \Rightarrow \#11111 \Rightarrow 1\#111 \Rightarrow 11\#1 $\\
6 ) $\epsilon \Rightarrow \# $\\

\textbf{(К) Дублирование всех символов входного слова (над алфавитом
$\Sigma = \{a, b \}$). Например: $abab \rightarrow aabbaabb$.}\\\\
Задан алфавит $\Sigma' = \{a, b \} \bigcap \{*\}$\\\\
$ \left\{
\begin{aligned}
& *a \rightarrow aa* \\
& *b \rightarrow bb* \\
& * \rightarrow .\epsilon \\
& \epsilon  \rightarrow  * \\
\end{aligned}
\right.$
\\\\
Примеры:\\
1 ) $a \Rightarrow *a \Rightarrow aa* \Rightarrow aa $\\
2 ) $ab \Rightarrow *ab \Rightarrow aa*b \Rightarrow aabb* \Rightarrow aabb $\\
3 ) $\epsilon \Rightarrow * \Rightarrow \epsilon $\\

\textbf{(Л)  Перестановка символов входного слова в обратном порядке (над алфавитом $\Sigma = \{a, b \}$).}\\\\
Задан алфавит $\Sigma' = \{a, b \} \bigcap \{\#, *\}$\\\\
$ \left\{
\begin{aligned}
&*aa \rightarrow a*a\\
&*ab \rightarrow b*a\\
&*ba \rightarrow a*b\\
&*bb \rightarrow b*b\\
&*a\# \rightarrow \#a\\
&*b\# \rightarrow \#b\\
&*\# \rightarrow .\epsilon\\
&* \rightarrow \#\\
&\epsilon \rightarrow *\\
\end{aligned}
\right.$\\\\
Примеры:\\
1 ) $aaba \Rightarrow *aaba \Rightarrow a*aba \Rightarrow ab*aa \Rightarrow  aba*a \Rightarrow aba\#a \Rightarrow *aba\#a \Rightarrow b*aa\#a \Rightarrow  ba*a\#a \Rightarrow$ 

$ba\#aa \Rightarrow *ba\#aa \Rightarrow a*b\#aa \Rightarrow  a\#baa \Rightarrow *a\#baa \Rightarrow \#abaa \Rightarrow *\#abaa \Rightarrow  abaa$\\
2 ) $aaa \Rightarrow *aaa \Rightarrow a*aa \Rightarrow aa*a \Rightarrow aa\#a \Rightarrow *aa\#a \Rightarrow a*a\#a \Rightarrow a\#aa \Rightarrow *a\#aa \Rightarrow \#aaa \Rightarrow$

$*\#aaa \Rightarrow aaa$\\
3 ) $\epsilon \Rightarrow * \Rightarrow \# \Rightarrow *\# \Rightarrow\epsilon$\\


\textbf{(М) Cортировка символов входного слова (над алфавитом
$\Sigma = \{a, b, c \}$).}\\\\
Задан алфавит $\Sigma = \{a, b, c \}$\\\\
$ \left\{
\begin{aligned}
&ba \rightarrow ab\\
&cb \rightarrow bc\\
&ca \rightarrow ac\\
\end{aligned}
\right.$
\\\\
Примеры:\\
1 ) $bcaacb\Rightarrow bacacb\Rightarrow abcacb\Rightarrow abaccb\Rightarrow aabccb\Rightarrow aabcbc\Rightarrow aabbcc$\\
2 ) $ccbbaa\Rightarrow cbcbaa\Rightarrow bccbaa\Rightarrow bcbcaa\Rightarrow bbccaa\Rightarrow bbcaca\Rightarrow bbacca\Rightarrow babcca\Rightarrow abbcca\Rightarrow$

$abbcac\Rightarrow abbacc\Rightarrow ababcc\Rightarrow aabbcc$\\

\textbf{(Н) Проверка, является ли входное слово палиндромом  (над алфавитом
$\Sigma = \{a, b \}$). Если является, то результатом должно быть пустое слово, если не является, то результатом может быть любое непустое слово.}\\\\
Задан алфавит $\Sigma' = \{a, b \} \bigcap \{*\}$\\\\
$ \left\{
\begin{aligned}
& *aa \rightarrow a*a \\
& *bb \rightarrow b*b \\
& *ab \rightarrow b*a \\
& *ba \rightarrow a*b \\
& a*a \rightarrow \epsilon \\
& b*b \rightarrow \epsilon \\
& a*b \rightarrow .a \\
& b*a \rightarrow .a \\
& *b \rightarrow .\epsilon \\
& *a \rightarrow .\epsilon \\
& * \rightarrow .a \\
& \epsilon \rightarrow * \\
\end{aligned}
\right.$
\\\\
Примеры:\\
1 ) $\epsilon \Rightarrow *\Rightarrow \epsilon$\\
2 ) $abba \Rightarrow *abba\Rightarrow b*aba\Rightarrow bb*aa\Rightarrow bba*a\Rightarrow bb\Rightarrow *bb\Rightarrow b*b\Rightarrow\epsilon$\\
3 ) $ababa\Rightarrow *ababa\Rightarrow b*aaba\Rightarrow ba*aba\Rightarrow bab*aa\Rightarrow baba*a\Rightarrow bab\Rightarrow *bab\Rightarrow a*bb\Rightarrow ab*b\Rightarrow$

$ a\Rightarrow *a\Rightarrow \epsilon$\\
4 )$aabb\Rightarrow *aabb\Rightarrow a*abb\Rightarrow ab*ab\Rightarrow abb*a\Rightarrow aba$ \\
5 ) $a\Rightarrow *a\Rightarrow \epsilon$\\

\textbf{(О) Проверка, является ли входное слово именем одного из основных регистров процессора Intel 8088 (AX, BX, CX или DX). Результатом должно быть либое имя регистра, либо пустое слово.}\\\\
Задан алфавит $\Sigma' = \{A,B,C,D,X \} \bigcap \{\#,|\}$\\

Рассмотрим схему для AX. Для остальных имен схема аналогична, но вместо $\#$ ставится другой элемент, для того что бы при возврате получить исходное имя.\\

Введем обозначение W - любая буква алфавита(A,B,C,D,X).\\

То есть запись вида $ W \rightarrow *$ обозначает следующую партию подстановок:
$ \left\{
\begin{aligned}
& A \rightarrow * \\
& B \rightarrow * \\
& C \rightarrow * \\
& D \rightarrow * \\
& X \rightarrow * \\
\end{aligned}
\right.$\\
СХЕМА ДЛЯ AX :\\
$ \left\{
\begin{aligned}
& AX \rightarrow \# \\
& W\#  \rightarrow \epsilon \\
& \#W \rightarrow \epsilon \\
& W \rightarrow \epsilon \\
& \#\# \rightarrow | \\
& |\# \rightarrow \epsilon \\
& | \rightarrow \epsilon \\
& \# \rightarrow .AX \\
\end{aligned}
\right.$
\\\\


\end{document}